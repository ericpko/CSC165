% === CSC165 Winter 2018 ===
% __authors__ = 'Jacob Chmura, Conor Vedova, Eric Koehli'
% === Problem Set 3 ===

% ******************************************************************************
%                              BEGIN PREAMBLE
% ******************************************************************************

\documentclass[12pt, a4paper]{amsart}

\usepackage{amssymb, amscd, amsmath, amsthm, url, microtype, enumitem}
\usepackage[margin=2.5cm]{geometry}
\usepackage{csc}
\usepackage{setspace} % A package for spacing (double space, etc.)
% \linespread{1.3} % halfspaced

\pagestyle{plain} % Avoids repeating the authors on every page.
\onehalfspacing   % part of setspace package
% \vspace{5mm} % adds a blank line.

% Either use these or the geometry package:
% \addtolength{\oddsidemargin}{-0.5in}
% \addtolength{\evensidemargin}{-0.5in}
% \addtolength{\textwidth}{1in}
% \addtolength{\topmargin}{-0.3in}
% \addtolength{\textheight}{0.3in}

\newtheorem {Theorem}                    {Theorem}
\newtheorem {Proposition}[Theorem]       {Proposition}
\newtheorem {Lemma}      [Theorem]       {Lemma}
\newtheorem {Corollary}  [Theorem]       {Corollary}
\newtheorem*{Claim}       {Claim}
\newtheorem*{Question}   {Question}
\newtheorem*{Exercise}   {Exercise}

\theoremstyle{definition}
\newtheorem  {Definition} [Theorem]{Definition}

\theoremstyle{remark}
\newtheorem*{Remark}	{Remark}
\newtheorem{Example}[Theorem]	{Example}

\numberwithin{Theorem}{section}


\newcommand{\mute}[1] {}
\newcommand{\mc}[1]{\mathcal{#1}}
\newcommand{\mf}[1]{\mathfrak{#1}}
\newcommand{\cal}[1]{\mathcal{#1}}

% \def \N{\mathbb{N}}
% \def \Z{\mathbb{Z}}
% \def \R{\mathbb{R}}
% \def \C{\mathbb{C}}
% \def \Q{\mathbb{Q}}

\def \vphi{\varphi}
\def \eps{\varepsilon}
\def \ssm{\smallsetminus}

% ******************************************************************************
%                               END PREAMBLE
% ******************************************************************************
% Document starts here
\begin{document}

% Document metadata
\title{CSC165H1 Winter 2018: Problem Set 3}
\author{By: Jacob Chmura, Conor Vedova, Eric Koehli}
\date{March 14, 2018}
\maketitle
\newpage
% End


\section{Special numbers}
define $F_n = 2^{2^{n}} + 1$. Prove that $$\forall n \IN \N, 
F_n - 2 = \prod_{i=0}^{n-1} F_i$$
From the definition above, it follows that $(1):$ $F_n - 2 = 2^{2^n} + 1 - 2 = 2^{2^n} - 1$.

\begin{proof}

Let $P(n)$ be the statement $F_n - 2 = \prod_{i=0}^{n-1} F_i$, where $n \IN \N$.
We will show that $P(n)$ is true for all natural numbers by induction on $n$.

\base 

Let $n = 0$. Then,

\begin{align*}
    F_0 - 2 &= 2^{2^0} + 1 - 2 \\
    &= 2^1 - 1 \\
    &= 1
\end{align*}

On the other hand, when a products lower bound is greater than it's upper bound,
the product is empty. Therefore,

\begin{align*}
    \prod_{i=0}^{0-1} F_i &= 1 = F_0 - 2
\end{align*}

This shows that $P(0)$ is true.

\istep

Let $k \IN \N$ and assume that $P(k)$ is true. That is, $\forall k \IN \N, 
F_k - 2 = \prod_{i=0}^{k-1} F_i$. We want to show that $P(k + 1)$ is true, that is,
$$F_{k+1} - 2 = \prod_{i=0}^{k} F_{i}$$

From the LHS we have,

\begin{align*}
    F_{k+1} - 2 &= 2^{2^{k+1}} + 1 - 2 \\
    &= \big( 2^{2^k} \big)^2 - 1 \\
    &= \big( 2^{2^k} - 1 \big) \big( 2^{2^k} + 1 \big) 
    \tag{$a^2 - b^2 = (a - b)(a + b)$} \\
    &= \big( F_k - 2 \big) \big( 2^{2^k} + 1 \big) \tag{By (1)} \\
    &= \bigg( \prod_{i=0}^{k-1}F_i \bigg) \cdot F_k \tag{By the induction hypothesis} \\
    &= \prod_{i=0}^{k}F_{i}
\end{align*}

Thus $P(k + 1)$ follows from $P(k)$, and this completes the induction step. Having
proved steps 1 and 2, we can conclude by the Axiom of Induction
that $P(n)$ is true for all natural numbers $n$.

\end{proof}

\newpage


\section{Sequences}

Define $a_0 = 1$, and for all natural numbers $n$, $$a_{n+1} = \frac{1}{\frac{1}{a_n} + 1}$$

\begin{enumerate}

    \item[(a)] $a_0 = 1$, $a_1 = \frac{1}{2}$, $a_2 = \frac{1}{3}$, $a_3 = \frac{1}{4}$
    
    \item[(b)] Since the numerator of the sequence is constant, we can put $1$ in the
    numerator. As n increases, the denominator is increasing by one, and since we are 
    starting at $n = 0$, we can write $a_n = \frac{1}{n + 1}$.
    
    Prove that for all natural numbers $n$, $$a_n = \frac{1}{n + 1}$$
    
    \begin{proof}
    
    Let $P(n)$ be the statement $a_n = \frac{1}{n + 1}$.
    We will show that $P(n)$ is true for all natural numbers $n$.
    
    \base
    
    Let $n = 0$. Then $$a_0 = \frac{1}{0 + 1} = 1$$ 
    and $a_0 = 1$ by definition. This shows that $P(0)$ is true.
    
    \istep
    
    Let $k \IN \N$ and assume that $P(k)$ is true. That is, 
    for all $k \IN \N$, $a_k = \frac{1}{k + 1}$.
    We want to show that $P(k + 1)$ is true. That is, 
    $$a_{k+1} = \frac{1}{k + 2}$$
    Starting from the recursive definition, we have
    
    \begin{align*}
        a_{k+1} &= \frac{1}{\frac{1}{a_k} + 1} \\
        &= \frac{1}{\frac{1}{\frac{1}{k+1}} + 1} \tag{By the induction hypothesis} \\
        &= \frac{1}{k+1 + 1} \\
        &= \frac{1}{k+2}
    \end{align*}
    
    Thus $P(k + 1)$ follows from $P(k)$, and this completes the induction step. Having
    proved steps 1 and 2, we can conclude by the Axiom of Induction
    that $P(n)$ is true for all natural numbers $n$.
    
    \end{proof}

\end{enumerate}

\newpage

Define $a_{k, 0} = k$, and for all natural numbers $n$, 
$$a_{k, n + 1} = \frac{k}{\frac{1}{a_{k, n}} + 1}$$

\begin{enumerate}

    \item [(c)] $a_{2, 0} = 2$, $a_{2, 1} = \frac{4}{3}$, $a_{2, 2} = \frac{8}{7}$, 
     $a_{2, 3} = \frac{16}{15}$, and
     
     $a_{3, 0} = 3$, $a_{3, 1} = \frac{9}{4}$, $a_{3, 2} = \frac{27}{13}$, 
     $a_{3, 3} = \frac{81}{40}$
    
    \item[(d)] Prove that for all natural numbers $k$ and $n$, that,
    
    $$k > 1 \IMP a_{k, n} = \frac{k^{n+1}}{\sum_{i = 0}^{n} k^i}$$
    
    \begin{proof}
    Let $k \IN \N$. Assume that $k > 1$.
    Let $P(n)$ be the statement
    
    $$a_{k, n} = \frac{k^{n+1}}{\sum_{i = 0}^{n} k^i}$$ 
    
    where $n \in \N$. We will show that $P(n)$ is true for all natural numbers $k > 1, n$,
    by Mathematical Induction on $n$.
    
    \base
    
    Let $n = 0$. In this case, we have
    $$a_{k, 0} = \frac{k^{1}}{\sum_{i = 0}^{0} k^i} = \frac{k}{k^{0}} = k$$
    
    This is the same result as the definition of $a_{k, 0}$, so $P(0)$ is satisfied.

    \istep
    
    Let $t \IN \N$ and assume that $P(t)$ is true. That is, we assume
    $$a_{k, t} = \frac{k^{t+1}}{{\sum_{i = 0}^{t} k^i}}$$
    and we want to show that $P(t + 1)$ is true. That is,
    $$a_{k, t+1} = \frac{k^{t+2}}{{\sum_{i = 0}^{t+1} k^i}}$$
    
    By the recursive definition,
    
    \begin{align*}
        a_{k, t+1} &= \frac{k}{\frac{1}{a_{k, t}} + 1} \\
        &= \frac{k}{\frac{\sum_{i = 0}^{t} k^i}{k^{t+1}} + 1} \tag{By induction hypothesis} \\
        &= \frac{k}{\frac{\sum_{i = 0}^{t} k^i + k^{t+1}}{k^{t+1}}} \\
        &= \frac{k \cdot k^{t+1}}{{\sum_{i = 0}^{t} k^i + k^{t+1}}} \\
        &= \frac{k^{t+2}}{{\sum_{i = 0}^{t+1} k^i}} 
    \end{align*}
    
    Thus $P(t + 1)$ follows from $P(t)$, and this completes the induction step. 
    Since we fixed $k$ to be an arbitrary natural number greater than $1$ prior to the induction, 
    it must be the case that the result holds for all natural numbers $k$ greater than $1$.
    Having proved steps 1 and 2, we can now conclude by the Axiom of Induction
    that $P(n)$ is true for all natural numbers $k > 1$ and $n$.
    
    \end{proof}

\end{enumerate}

\newpage


\section{Properties of Asymptotic Notation}

\begin{enumerate}
    
    \item[(a)] Define $Sum_f(n) = \sum_{i = 0}^{n} f(i) = f(0) + f(1) + \dots + f(n)$.
    
    Prove that for all $f$: $\N \rightarrow \R^{\geqslant 0}, f \in O(n) \implies Sum_f \in O(n^2)$
    
    \begin{proof}
    
    Let $f$: $\N \rightarrow \R^{\geq 0}$ and 
    assume that $f \in O(n)$. That is, \\
    $\exists c_0, n_0 \in \R^{+}, \forall n \in \N, n \geq n_0 \implies f(n) \leq c_0 \cdot n$
    
    Want to show that the sum of the arbitrary function $f$ is in $O(n^2)$. That is,
    $Sum_f (n) \in O(n^2)$. Or, in expanded form, \\
    $\exists c_1, n_1 \in \R^{+}, \forall n \in \N, n \geq n_1 \implies Sum_f(n) \leq c_1 \cdot n^2$
    
    \medbreak
    
    Let $c_0, n_0$ be such that $f \IN O(n)$ for these values. Let $c_1 = \sum_{i = 0}^{n_0 - 1} f(i) + c_0$ and $n_1 = \max(n_0, 1)$. Let $n \IN \N$ and assume $n \geqslant n_1$.
    We know that $n \geq n_1 \geq n_0$. So, it follows that $f(n) \leq c \cdot n$. Using this, let's analyze the following sum:
    
    \begin{align*}
        \sum_{i = 0}^{n} f(i) &= \sum_{i = 0}^{n_0 - 1} f(i) + \sum_{i =  n_0}^{n} f(i) \\
        &\leq \sum_{i = 0}^{n_0 - 1} f(i) + \sum_{i =  n_0}^{n} c_0n\tag{$\forall n \geq n_0, f(n) \leq c_0n$} \\
        &= \sum_{i = 0}^{n_0 - 1} f(i) + (n-n_0) \cdot c_0n \\
        &= \sum_{i = 0}^{n_0 - 1} f(i) - c_0nn_0 + c_0n^2 \\
    \end{align*}
    Since for any $k \IN \R^{\geqslant 0}$ and $n \geq 1$, $k \leq k \cdot n^2$, which holds by the way $n_1$ was chosen, 
    \begin{align*}
         \sum_{i = 0}^{n_0 - 1} f(i) &\leq (\sum_{i = 0}^{n_0 - 1} f(i)) \cdot n^2 \tag{Because $f(n) \geq 0$}
    \end{align*}
    So,
    \begin{align*}
        \sum_{i = 0}^{n_0 - 1} f(i) - c_0nn_0 + c_0n^2 &\leq (\sum_{i = 0}^{n_0 - 1} f(i)) \cdot n^2 - c_0nn_0 + c_0n^2 \\
        &\leq (\sum_{i = 0}^{n_0 - 1} f(i)) \cdot n^2 + c_0n^2 \tag{since $-c_0nn_0 \leq 0$} \\
        &= (\sum_{i = 0}^{n_0 - 1} f(i)) + c_0) \cdot n^2 \\
        &= c_1 \cdot n^2
    \end{align*}
    So, it follows that $Sum_f(n) \leq c_1 \cdot n^2$. Therefore, $Sum_f(n) \IN O(n^2)$
   
    \end{proof}
    
    \item[(b)] Prove that for all natural numbers $n$,
    $$\sum_{i = 1}^{2^n} \frac{1}{i} \geq \frac{n}{2}$$ 
     
    \begin{proof}
    
    Let $P(n)$ be the statement above, where $n \in \N$. 
    We will prove that $P(n)$ is true for all natural numbers by mathematical induction.
     
    \base 
     
    Let $n = 0$.
    Then we have,
    $$\sum_{i = 1}^{2^0} \frac{1}{i} = \sum_{i = 1}^{1} \frac{1}{i} = 1 \geq \frac{0}{2} = \frac{n}{2}$$
    This shows that $P(0)$ is true.
     
    \istep
     
    Let $k \in \N$ and assume that $P(k)$ is true. That is, 
    $$\sum_{i = 1}^{2^k} \frac{1}{i} \geq \frac{k}{2}$$
    We want to show that $P(k+1)$ is true. That is,
    $$\sum_{i = 1}^{2^{k+1}} \frac{1}{i} \geq \frac{k+1}{2}$$

    Analyzing the left hand side,
    
    \begin{align*}
        \sum_{i=1}^{2^{k+1}} \frac{1}{i} &= \sum_{i=1}^{2^k} \frac{1}{i} + \frac{1}{2^k + 1} + \frac{1}{2^k + 2} + ... + \frac{1}{2^k + 2^k} \\
        &\geq \frac{k}{2}+ \frac{1}{2^k + 1} + \frac{1}{2^k + 2} + ... + \frac{1}{2^k + 2^k} \tag{By induction Hypothesis} \\
    \end{align*}
    
    Now to proceed with the proof, it will be shown that $\frac{1}{2^k + 1} + \frac{1}{2^k + 2} + ... + \frac{1}{2^k + 2^k} \geq \frac{1}{2}$.
    
    Notice,  $\frac{1}{2^k + 1} + \frac{1}{2^k + 2} + ... + \frac{1}{2^k + 2^k} \geq \frac{1}{2^k + 2^k} + \frac{1}{2^k + 2^k} + ... + \frac{1}{2^k + 2^k}$, since $\frac{1}{2^k + k}$ is the smallest element of the summation. 
    
 \newpage
    
    With this fact in mind:
    
    \begin{align*}
        \sum_{i=1}^{2^{k+1}} &\geq \frac{k}{2}+ \frac{1}{2^k + 1} + \frac{1}{2^k + 2} + ... + \frac{1}{2^k + 2^k} \\
        &\geq \frac{k}{2} + \frac{1}{2^k + 2^k} + \frac{1}{2^k + 2^k} + ... + \frac{1}{2^k + 2^k} \\
        &= \frac{k}{2} + 2^k \cdot \frac{1}{2^k + 2^k}\\
        &= \frac{k}{2} + \frac{1(2^k)}{2 \cdot 2^k} \\
        &= \frac{k}{2} + \frac{1}{2} \\
        &= \frac{k+1}{2}
    \end{align*}
    
    Thus $P(k + 1)$ follows from $P(k)$, and this completes the induction step. Having
    proved steps 1 and 2, we can conclude by the Principle of Mathematical Induction
    that $P(n)$ is true for all natural numbers $n$.
    \end{proof}
     
    \newpage
     
     \item[(c)]
     \emph{Disprove} The following claim: \\
     $\forall f, g: \N \xrightarrow{} \R^{\geq 0}, f(n) \in O(g(n)) \implies Sum_f(n) \in O(n\cdot g(n))$. \\
     
     We will prove the negation of the statement: \\
     $\exists f, g: \N \xrightarrow{} \R^{\geq 0}, f(n) \in O(g(n)) \land Sum_f(n) \notin O(n\cdot g(n))$ \\
     
    \begin{proof} 
    Let $f(n) = \frac{1}{n}$ and $g(n) = \frac{1}{2 \cdot n}$ \\
    
    First showing $f \in O(g)$: \\
     $\exists c_0, n_0 \in \R^{+}, \forall n \in \N, n \geq n_0 \implies \frac{1}{n} \leq c_0 \cdot \frac{1}{2 \cdot n}$
    
    Let $n_0 = 0$ and $c_0 = 2$
    Let $n \in \N$.
    Assume $n \geq n_0$
    Then $\forall n$:
    
\begin{align*}
    f(n) &= \frac{1}{n} \\
    &\leq \frac{1}{n} \\
    &= 2 \cdot \frac{1}{2 \cdot n} \\
    &= c_0 \cdot g(n)
\end{align*} \\
    
    So by definition,  $f \in O(g)$.
    
    Now we will show the following by negating the definition of Big-Oh:
    
    $Sum_f(n) \notin O(n\cdot g(n))$:
    
    $\forall c_1, n_1 \in \R^{\geq 0}, \exists n \in \N, n \geq n_1 \land Sum_f (n) > c_1 \cdot n \cdot g(n)$ \\
    
    
    Let $c_1, n_1 \in \R^{\geq 0}$ \\
    Take $n = max(2^{\ceil {c_1}} + 2, 2^{n_1})$
    Then it follows that $n \geq 2^{n_1} \geq n_1$, and the first part of the \emph{and} is satisfied.
    
    Also, $n > 2^{\ceil{c_1}}  \implies log_2 n > \ceil{c_1} \geq c_1$. So $log_2 n > c_1 $(1)
    
    Importantly, by the way that $n$ was chosen, it must be a power of two.
    It follows that $log_2 n \in \N$. (2)
    
    
    From (b) we know that for all natural numbers $t$:
    
    $$\sum_{i = 1}^{2^t} \frac{1}{i} \geq \frac{t}{2}$$
    
    Change of variable: Let $t = log_2 n$. By (2) we know that $t \in \N$ and we have:
    
    $$\sum_{i = 1}^{n} \frac{1}{i} \geq \frac{log_2 n }{2}$$
    
    Now we can see that the left side of the inequality is just $Sum_f (n)$, so it follows that:
\begin{align*}
    Sum_f (n)  &\geq \frac{log_2 n }{2} \\
    &= g(n)\cdot n \cdot log_2 n \\
    &> g(n) \cdot n \cdot c_1 \tag{(By (1), $log_2 n > c_1$}
\end{align*}

So, $n \geq n_1 \land Sum_f(n) > c_1 \cdot n \cdot g(n)$. 

Therefore, $Sum_f(n) \notin O(n\cdot g(n))$. \\

Together, we have proven $\exists f, g: \N \xrightarrow{} \R^{\geq 0}, f(n) \in O(g(n)) \land Sum_f(n) \notin O(n\cdot g(n))$. Since the negation of the original is proven, the original statement is disproven.
\end{proof}
\end{enumerate}

\end{document} % The document ends here
